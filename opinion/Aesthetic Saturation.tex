\documentclass[12pt]{article}
\usepackage{geometry}
\usepackage{setspace}
\usepackage{lmodern}
\usepackage[T1]{fontenc}
\usepackage{amsmath,amssymb}
\usepackage{microtype}

\geometry{margin=1in}
\onehalfspacing

\begin{document}

\title{Aesthetic Saturation and the Erosion of Voluntary Attention in Algorithmic Social Media}

\author{Flyxion}
\date{\today}

\maketitle

\begin{abstract}
This essay analyzes the contemporary design and behavior of large-scale social media platforms through the lens of aesthetic degradation, coerced expressivity, and attention collapse. Focusing on the experiential failures of algorithmically curated feeds, it argues that visual tackiness, prompt saturation, enforced performativity, and unsolicited content intrusion are not accidental byproducts of poor taste, but necessary consequences of optimization regimes that treat attention as an undifferentiated resource. The analysis situates avatars, stickers, engagement prompts, follow limits, and genre intrusion as mechanisms that systematically erase silence, refusal, and long-horizon curiosity as legitimate user states. While these features generate irritation at the surface level, their deeper effect is epistemic and temporal: the dissolution of coherence, the abolition of duration, and the replacement of meaning with metric-driven motion. The essay concludes by framing refusal, restraint, and academic articulation as counter-technologies that reintroduce constraint, gravity, and intelligibility into environments designed to resist all three.
\end{abstract}

\newpage
\section{Introduction}

The experience of large-scale social media platforms increasingly resists description through ordinary evaluative language. Familiar terms such as distraction, annoyance, or excess gesture toward surface effects, but fail to capture a deeper systemic failure unfolding simultaneously at aesthetic, cognitive, and epistemic levels. What confronts the user is not merely an abundance of low-quality material, but a systematic hostility to coherence itself. The interface does not simply present content; it actively destabilizes the conditions under which content might be interpreted, remembered, or refused.

This essay takes as its starting point the observation that ugliness on such platforms is not incidental. The gaudiness of visual design, the redundancy of prompts, the insistence on exaggerated expressivity, and the collapse of contextual boundaries together form a recognizable pattern. These features persist not because they are overlooked, but because they are functionally aligned with an underlying optimization regime that privileges measurable engagement over intelligibility, duration, or taste. The resulting environment appears chaotic, yet it is internally consistent with objectives that treat attention as a resource to be agitated rather than a faculty to be respected.

The analysis proceeds from lived phenomenology rather than abstract critique. It attends to the experience of being addressed repeatedly by interfaces that refuse silence, to the irritation of expressive demands that cannot be dismissed, and to the exhaustion produced by constant unsolicited intrusion. Such experiences are often dismissed as matters of personal preference or generational temperament. This essay rejects that framing. The reactions described here are not idiosyncratic but structurally induced, arising wherever systems erase negative space and interpret refusal as failure.

Central to the argument is the claim that these platforms have inverted the relationship between interface and content. Where earlier media sought, however imperfectly, to withdraw in service of what they displayed, contemporary feeds compete aggressively for perceptual dominance. Expression becomes performative rather than communicative, visibility becomes correlated with artifice, and silence is rendered illegible. Under these conditions, even sincere participation risks being absorbed into the very mechanisms that degrade it.

By situating these failures within a broader account of algorithmic saturation and metric-driven design, the essay aims to clarify why dissatisfaction with such platforms feels both personal and inescapable. It is not simply that the content is bad, but that the system has been engineered to prevent the emergence of better forms. In articulating this condition with precision and continuity, the essay seeks not merely to complain, but to reassert the legitimacy of restraint, refusal, and coherence as design constraints that have been systematically abandoned.


\section{Aesthetic Degradation as Systemic Signal}

The contemporary interface of \emph{Facebook} may be analyzed not merely as a failure of taste, but as a visible symptom of deeper systemic decay. What presents itself phenomenologically as tackiness, gaudiness, or visual noise is not accidental ornamentation, nor even the cumulative result of poor design decisions, but rather the necessary surface manifestation of an optimization regime that no longer recognizes aesthetic coherence as a constraint. The platform's visual field has become saturated with fillers, redundant prompts, artificial projections of engagement, and aggressively emotive cues, none of which are required for informational function, yet all of which are indispensable to sustaining its internal metrics.

Historically, interface minimalism served as an implicit contract between system and user: the platform would recede into the background, allowing content to assert primacy. That contract has been decisively broken. The proliferation of decorative avatars, animated stickers, incessant call-to-action bubbles, and emoji-laden captions signals a reversal of priorities in which the interface no longer supports content but competes with it. The result is an aesthetic environment that resembles neither a forum nor a publication, but a continuously mutating advertisement for its own affordances.

This aesthetic degradation is not neutral. Visual incoherence functions as a regulatory mechanism, reducing the user's capacity for sustained attention while simultaneously increasing susceptibility to low-effort stimuli. Excessive ornamentation, predictive filler text, and speculative pseudo-content create an environment in which discernment becomes cognitively expensive. Under such conditions, the platform benefits not from clarity but from saturation, as confusion prolongs dwell time even when it diminishes satisfaction.

The grotesque persistence of avatar prompts exemplifies this logic. Their repetition is not evidence of a malfunction but of a system indifferent to refusal. The repeated invitation to ``create your own avatar,'' even after years of disinterest, illustrates a design paradigm in which negative feedback is not interpreted as signal but ignored as noise. Refusal is not a state the system recognizes; it is merely a temporary failure to convert.

Thus, the ugliness of the interface should not be mistaken for incompetence. It is the aesthetic signature of a platform that has fully externalized the costs of annoyance, distraction, and visual fatigue onto its users while internalizing the benefits of behavioral compliance. What appears as chaos is, in fact, a stable equilibrium under an objective function hostile to taste, restraint, and silence.

\section{Avatars, Stickers, and the Coercion of Performativity}

The introduction and subsequent entrenchment of avatars and animated stickers marks a decisive shift from voluntary expression to enforced performativity. These elements do not merely offer additional modes of self-representation; they function as compulsory intermediaries through which interaction is increasingly expected to pass. Their presence alters the ontology of participation itself, transforming silence, restraint, or textual precision into anomalous behaviors that the system persistently attempts to correct.

Avatars are especially revealing in this regard. Ostensibly framed as tools for personalization, they instead produce a homogenized field of expression whose uncanny affect derives precisely from its attempt to simulate individuality through constrained templates. The resulting figures are neither abstract nor human, but occupy an unsettling middle ground that invites projection while resisting identification. Their creepiness is not incidental. It emerges from the tension between imposed cheerfulness and the absence of genuine intentionality, a tension exacerbated by their repetitive deployment across unrelated contexts.

Stickers amplify this problem by collapsing communication into prefabricated emotional tokens. Where language permits gradation, hesitation, and ambiguity, stickers impose discrete affective states that are instantly legible to the system and easily counted. The mandatory textual footers reveal the true function of these artifacts. Each sticker is not merely an utterance but an advertisement for further compliance, a recursive prompt embedded directly within social interaction. Expression is thus never allowed to terminate; it must always point back toward deeper integration with the platform's representational machinery.

The refusal of avatars and stickers is therefore not a rejection of playfulness or creativity, but a defense of negative expressive space. The ability not to emote, not to decorate, and not to signal engagement is a precondition for meaningful participation in any communicative environment. Facebook's persistent reintroduction of these elements despite clear disinterest indicates that the system does not model user preference as a legitimate constraint. Instead, it treats resistance as an optimization problem to be overcome through repetition.

This coercion produces a paradoxical outcome. By forcing expressivity everywhere, the platform renders expression itself meaningless. When every interaction is padded with exaggerated affect and visual noise, the distinction between signal and filler collapses. Communication becomes performative without being expressive, visible without being informative, and animated without being alive. What remains is not community, but a continuous pantomime staged for metrics that mistake motion for meaning.

\section{Prompt Saturation and the Pathology of Unignorable Interfaces}

Beyond overt visual clutter, the platform exhibits a more insidious failure mode in its relentless deployment of prompts that cannot be meaningfully dismissed. These include notification bubbles, profile overlays, transient input fields, and recurrent invitations to ``share a thought,'' each of which presumes that expression is both desired and overdue. The irritation these prompts generate does not arise merely from their frequency, but from their structural refusal to acknowledge prior rejection.

The appearance of a speech bubble hovering over one's own profile image is emblematic of this pathology. It transforms the user's presence into an incomplete task, rendering silence as an error state requiring correction. The interface thus ceases to be descriptive and becomes accusatory. It no longer reflects what the user is doing, but insists upon what the user ought to be doing according to the system's internal schedule of engagement.

What is especially revealing is the ephemerality of compliance. When the user briefly acquiesces, entering a mathematical remark or a minimal utterance, the system does not interpret this as closure. The content expires, vanishes, and is promptly replaced by the same demand that occasioned it. Expression is therefore treated not as communication but as a consumable placeholder, valuable only insofar as it resets a timer. The system does not remember that it has been answered, because memory would introduce obligation, and obligation would constrain future prompting.

This design pattern produces a peculiar inversion of authorship. The user is nominally the speaker, yet the cadence, visibility, and lifespan of the utterance are entirely controlled by the platform. Speech is solicited, displayed briefly, and then erased, not because it lacks value, but because permanence would reduce the system's leverage. What matters is not what is said, but that something is said often enough to maintain a surface impression of vitality.

Such prompt saturation undermines the basic communicative norm that silence can be intentional. In most human contexts, not speaking conveys as much information as speaking, particularly when abstention reflects discernment or boundary-setting. Facebook's interface architecture systematically erases this distinction. Silence is not allowed to signify refusal, contemplation, or completion; it can only mean neglect. The user is perpetually framed as behind, inactive, or incomplete.

The cumulative effect is a form of low-grade coercion that erodes trust in the interface itself. When every visible affordance is a demand rather than an option, the interface ceases to function as a tool and becomes an adversary. Interaction is no longer initiated by interest or intent, but by the need to suppress irritation. Under these conditions, even genuine moments of expression become tainted by the suspicion that they were extracted rather than chosen.

\section{Quantitative Limits and the Illusion of Abundance}

The imposition of rigid quantitative limits on following behavior exposes a deep contradiction in the platform's self-presentation as an arena of open discovery. In principle, the act of following serves as a low-cost signal of curiosity rather than endorsement, allowing users to maintain a wide peripheral awareness of potential sources of value. In practice, the rapid exhaustion of follow capacity transforms this exploratory gesture into a scarce resource that must be rationed prematurely, long before meaningful signal can plausibly emerge.

This scarcity is especially perverse given the platform's scale. When millions of pages exist, the likelihood that high-quality, non-sensational content appears early in a random exploration process is vanishingly small. Imposing follow limits in such an environment effectively biases discovery toward early, popular, or aggressively optimized pages, reinforcing the dominance of those already adept at gaming engagement metrics. The platform thereby conflates curiosity with commitment, penalizing long-horizon interest in favor of short-term performative allegiance.

The contrast with systems such as \emph{GitHub} is instructive. There, following is explicitly decoupled from immediacy and reward. One may follow orders of magnitude more projects than will ever be actively consumed, precisely because the system acknowledges that value often emerges slowly, unpredictably, and asynchronously. The follow graph in such contexts functions as a speculative archive rather than an attention funnel, preserving optionality without imposing artificial urgency.

Facebook's limits reveal that it does not conceptualize following as an epistemic act. Instead, it treats it as a lever of engagement density, a mechanism for concentrating attention rather than distributing it. Once the follow ceiling is reached, the user is forced into a false choice between abandoning exploratory breadth and engaging in aggressive pruning. This pruning, in turn, encourages blocking at scale, not as a principled stance but as a defensive necessity.

The irony is that such limits coexist with an effectively unbounded inflow of advertisements and unsolicited content. Blocking ninety-nine percent of ads does not appreciably reduce their presence, indicating that suppression is modeled as a local inconvenience rather than a global signal. The system registers the block but declines to infer preference, because inference would threaten the invariance of the revenue stream.

In this configuration, abundance is performative while scarcity is real. The user is inundated with material they did not request, yet constrained in their ability to express long-term curiosity about material that does not yet exist. The resulting environment is not one of discovery but of attrition, in which exploration is gradually replaced by defensive curation and disengagement becomes the only scalable response.

\section{Algorithmic Intrusion and the Collapse of Contextual Boundaries}

The most corrosive feature of the platform's current operation is the systematic erosion of contextual boundaries through algorithmic intrusion. Content no longer arrives as a consequence of deliberate subscription or expressed interest, but as an unsolicited assertion of relevance. Reels from unfollowed pages, sports highlights, gambling promotions, frenetic animal clips, and politically charged fragments appear without invitation, unified only by their capacity to provoke rapid affective response. The user's stated preferences are not consulted; they are overridden.

This intrusion is not merely excessive but categorical. Distinct domains of attention that would ordinarily be segregated by intent are forcibly collapsed into a single feed. Leisure, information, solicitation, spectacle, and agitation are blended into an undifferentiated stream, depriving the user of the ability to modulate cognitive mode. The absence of stable contextual cues ensures that no sustained interpretive stance can be maintained. One is perpetually interrupted, not by novelty in itself, but by novelty of the wrong kind at the wrong time.

The desire to disable entire classes of content reveals an implicit critique of the platform's epistemology. Sports, gambling, viral animals, and high-arousal political material are not objectionable because they exist, but because they are injected into spaces where they do not belong. Their forced proximity to unrelated material degrades all categories simultaneously. What might have been tolerable within a bounded channel becomes intolerable when imposed as ambient noise.

The platform's refusal to offer granular exclusion mechanisms is telling. To allow users to permanently suppress entire genres would require acknowledging that attention is structured, finite, and purpose-dependent. Instead, the system treats attention as an undifferentiated reservoir to be continuously agitated. Calm, focus, and low-arousal states are interpreted as underutilized capacity rather than legitimate goals.

This design choice creates a constant background of frenetic energy that is experienced not as stimulation but as agitation. The feed becomes an engine of interruption, optimized to prevent the formation of sustained cognitive trajectories. Over time, this produces a learned aversion not merely to specific content types, but to the platform itself. The user does not feel overwhelmed by information, but by its irrelevance, its mistiming, and its indifference to context.

In collapsing all boundaries, the platform ultimately collapses meaning. When everything demands attention simultaneously, nothing can hold it. What remains is a state of low-grade irritation punctuated by momentary spikes of engagement, a pattern that maximizes measurable interaction while steadily eroding the conditions under which genuine interest might arise.

\section{Tackiness as a Diagnostic for Automated Content}

Despite the pervasive dysfunction of the platform, its excesses inadvertently provide a crude but effective diagnostic for low-integrity content. The very features that render the environment unbearable also make certain pathologies unmistakable. Pages that post porn-adjacent imagery, algorithmic gratitude messages, follower milestone celebrations, engagement checklists, or auto-generated affirmations reliably announce themselves through a distinctive aesthetic signature. Their content is not merely uninteresting; it is structurally vacant.

These signals are valuable precisely because they are difficult to counterfeit in the opposite direction. While thoughtful, restrained, or technically substantive content can be drowned out by noise, automated or scam-driven material tends to overperform its own artificiality. Excessive emoji usage, templated enthusiasm, and self-referential announcements betray the absence of an underlying communicative goal. The content exists solely to be seen, not to say anything, and its form reflects this emptiness.

In this sense, tackiness functions as an unintended honesty mechanism. The platform's optimization for visibility rewards behaviors that are incompatible with subtlety, resulting in a feedback loop in which low-effort automation becomes increasingly garish. The system does not merely permit this; it amplifies it. Pages thanking followers for reaching arbitrary numeric thresholds or posting screenshots of engagement advice are effectively rewarded for revealing their own vacuity.

This phenomenon mirrors older print media pathologies, particularly in magazines saturated with advertisements. Yet the comparison ultimately flatters the earlier form. Traditional magazines, however cluttered, maintained at least a minimal editorial spine. Advertisements were bounded, identifiable, and subordinate to an organizing aesthetic. On Facebook, by contrast, the distinction between content and solicitation dissolves entirely. Everything is promotional, including posts that purport to be personal or informational.

The result is an environment that paradoxically trains users to distrust visibility itself. High exposure becomes correlated not with value, but with artifice. The user learns to interpret enthusiasm, repetition, and self-congratulation as warning signs rather than invitations. While this adaptive skepticism offers some protection, it is a poor substitute for genuine curation. It places the burden of discernment entirely on the user, who must continuously filter for integrity in a field optimized to suppress it.

Thus, even the platform's accidental affordances reinforce a grim conclusion. When ugliness becomes a reliable indicator of automation and scam, it ceases to be merely an aesthetic failure and becomes an epistemic one. The system trains its users not to seek excellence, but to avoid the worst, a posture that inevitably narrows the horizon of what can be discovered at all.

\section{From Magazine Clutter to Algorithmic Vandalism}

The comparison to magazines, though initially tempting, ultimately underscores the unprecedented nature of the platform's failure. Print magazines, even at their most commercial, were constrained by material costs, editorial sequencing, and the physics of the page. Advertisements competed for limited space and were therefore forced to coexist with content under a shared aesthetic regime. Excess could be irritating, but it remained legible. One could flip past it, ignore it, or contextualize it within a stable layout.

Facebook inherits none of these constraints. Its feed is effectively infinite, its layout perpetually mutable, and its ordering governed by opaque heuristics rather than editorial judgment. What replaces curation is not neutrality but vandalism by algorithm. Content is not arranged to be read, but to be reacted to, and coherence is actively disincentivized because it reduces interruptive potential. Where a magazine page accumulates clutter gradually, the feed delivers it instantaneously and without hierarchy.

The metaphor of a magazine curated by an entity without sensory grounding is therefore apt. The platform behaves as though it lacks any model of taste, restraint, or proportion. Yet this is not because such models are unavailable, but because they are incompatible with its objectives. Aesthetic coherence would impose limits on insertion frequency, visual intensity, and redundancy, all of which would reduce engagement volatility. The system instead embraces maximal heterogeneity, producing an experience that is simultaneously overstimulating and empty.

This degeneration is compounded by the platform's inability to forget. In print, once an issue passes, its clutter is gone. On Facebook, every failed design experiment persists in some form, layered atop previous ones. Avatars do not replace stickers; they accumulate. Prompts do not retire; they recur. Features introduced to address declining engagement remain active long after their novelty has expired, contributing to a sedimentary buildup of interface debris.

The resulting environment is not merely busy but hostile to narrative continuity. There is no beginning, middle, or end, only a perpetual present in which unrelated fragments jostle for dominance. Attention is fragmented not by accident but by design, as sustained trajectories are antithetical to the platform's measurement regime. What emerges is not a publication, nor even a forum, but a demolition site in which content is continuously smashed into clickable rubble.

In this context, the image of throwing spaghetti at a wall acquires unexpected dignity. At least gravity imposes coherence. The spaghetti falls, accumulates, and obeys physical constraints that allow the observer to predict and understand its behavior. The feed, by contrast, resists all such intuition. It drips without settling, splatters without pattern, and reforms endlessly, offering motion without structure and novelty without meaning. If the platform is engaging, it is only in the way a malfunctioning machine demands attention by refusing to stop.

\section{Toward Refusal and the Prospect of Better Forms}

What ultimately distinguishes the platform's failures from mere annoyance is their cumulative effect on the conditions of voluntary participation. The problem is not that Facebook is distracting, tasteless, or excessive in isolation, but that it systematically erodes the user's ability to opt out of those qualities without exiting the system entirely. One may block, mute, hide, or dismiss, yet these actions never converge toward quiet. They merely defer the next intrusion. Control is simulated locally while being denied globally.

This produces a distinctive form of fatigue that is neither informational overload nor moral outrage, but aesthetic and epistemic exhaustion. The user is not overwhelmed by meaning, but by its absence. The effort required to continuously negate unwanted stimuli becomes itself a drain on attention, transforming participation into a defensive activity. Over time, even the act of searching for worthwhile content begins to feel futile, as the surrounding noise negates the reward of discovery.

Refusal, under these conditions, becomes the only coherent stance. Not refusal in the dramatic sense of denunciation or exit, but a quieter, more structural refusal to internalize the platform's definitions of relevance, engagement, and expression. Blocking at scale, ignoring prompts, declining to emote, and withholding performative participation are all symptoms of this refusal. They are not expressions of disengagement from thought, but attempts to preserve it.

Yet refusal alone is not sufficient. The persistent thought that ``something better could be built'' arises precisely because the failures are not mysterious. The comparison to GitHub, to restrained print media, or even to the simple coherence of physical processes highlights that alternative design logics already exist. Systems can respect silence, allow long-horizon curiosity, separate domains of attention, and treat negative feedback as information rather than resistance to be worn down.

The tragedy of Facebook is therefore not that it is ugly, but that it did not have to be. Its current form is the result of deliberate choices to privilege metric volatility over legibility, compulsion over consent, and saturation over structure. The resulting experience feels less like communication and more like living inside an unending A/B test whose failures are never retired.

In this light, the desire to build something better is not utopian but corrective. It is an attempt to reintroduce constraint where it has been systematically removed, to restore boundaries where they have been collapsed, and to allow attention to settle rather than be perpetually agitated. Until such alternatives exist at scale, the most rational response may indeed be to look away. Even watching spaghetti obey gravity offers more coherence than a system that has forgotten how to stop.

\section{Coda: Coherence, Gravity, and the Right to Silence}

What lingers after sustained exposure to such an environment is not anger alone, but a sharpened awareness of what has been lost. Coherence, once broken, becomes conspicuous by its absence. One begins to notice how rare it is for systems to permit things to settle, to conclude, or to remain quiet without being treated as incomplete. The platform's greatest offense may be its inability to tolerate equilibrium. Everything must move, pulse, refresh, and solicit, even when no further action is warranted.

In physical systems, friction and gravity serve as organizing principles. They dissipate excess energy, prevent perpetual agitation, and allow structures to form. The digital environment under discussion appears to have been engineered to abolish these stabilizing forces. There is no friction to slow repetition, no gravity to pull features out of circulation once their usefulness has expired. Instead, novelty is simulated through incessant rearrangement, while entropy accumulates unchecked at the level of meaning.

The right to silence is thus not a sentimental preference but a structural necessity. Silence allows signals to differentiate, intentions to clarify, and value to emerge over time. When silence is pathologized, when every pause is treated as a deficit to be corrected, communication degenerates into noise. Expression ceases to be chosen and becomes merely induced. Under such conditions, even articulate protest risks being absorbed as just another data point.

To articulate these failures in academic prose is itself a minor act of resistance. It imposes order, continuity, and constraint on an experience designed to resist all three. It insists that the problem is not ineffable, not merely personal, and not reducible to nostalgia or temperament. The system behaves this way because it has been optimized to do so, and it will continue until different constraints are enforced.

If there is any consolation, it lies in the fact that such systems are historically contingent. They are not natural laws but assemblages of incentives, metrics, and design assumptions. Just as gravity makes spaghetti fall rather than hover indefinitely, alternative constraints could make digital spaces settle into forms that respect attention, refusal, and restraint. Until then, coherence will remain more readily observable in the behavior of matter than in the behavior of feeds, and stepping away will often be the only way to remember what intelligibility feels like.

\section{Afterword: Exit Without Escape}

There is a final asymmetry worth noting, one that explains why disengagement from such systems feels unsatisfying even when it is rational. Leaving does not restore the time or attention already consumed, nor does it repair the broader informational environment that continues to shape discourse elsewhere. The platform's influence persists as an externality, altering norms of expression, expectation, and pacing even for those who participate minimally. One exits the interface but not the culture it has helped deform.

This persistence produces a peculiar bind. Remaining invites irritation and aesthetic insult; leaving concedes terrain to the very dynamics one rejects. The system thus externalizes not only cognitive cost but moral burden, positioning users as individually responsible for managing a collective failure. To block, mute, and withdraw is framed as a personal preference rather than a rational response to structural hostility. The absence of viable, large-scale alternatives intensifies this bind, making refusal feel like retreat rather than progress.

What distinguishes the present moment from earlier media transitions is the collapse of temporal depth. Older systems, for all their flaws, allowed works to age, to recede, and to be rediscovered under different conditions. Here, nothing matures. Content is either immediately consumed or immediately buried, leaving no residue from which meaning might later be extracted. The present devours itself continuously, producing motion without accumulation and memory without history.

In this sense, the platform does not merely waste attention; it abolishes duration. It replaces the slow accrual of understanding with a regime of constant partial awareness, in which nothing is permitted to linger long enough to matter. The user's growing resentment is therefore not simply aesthetic or political, but temporal. It is the recognition that one's finite lifespan is being subdivided into fragments too small to hold thought.

To name this condition precisely is already to step partially outside it. Academic prose, with its insistence on continuity, argument, and closure, functions here as a counter-technology. It restores sequence where the feed enforces simultaneity and restores gravity where the interface demands perpetual suspension. Whether or not something better is built soon, the act of articulating these failures preserves the possibility that coherence remains intelligible and that silence, once reclaimed, can still mean something.

\newpage
\section*{Appendices}

\appendix
\section{Structural Model}

Let the platform be represented as a discrete-time dynamical system
\[
\mathcal{P} = (U, C, I, A, R)
\]
where \(U\) is the set of users, \(C\) the set of content items, \(I\) the interface state, \(A\) the action space, and \(R\) a ranking functional.

At each time step \(t\), the interface presents a feed
\[
F_t = R(U, C, I_t)
\]
with no requirement that \(C\) be restricted to content explicitly selected by the user.

Let \(A_u \subset A\) denote the admissible user actions, including posting, reacting, following, blocking, and ignoring. Define a subset \(A_u^{\mathrm{ref}}\) corresponding to refusals, including non-response and dismissal.

Assume the update rule
\[
I_{t+1} = \Phi(I_t, A_u, F_t)
\]
does not admit a fixed point corresponding to persistent refusal; that is,
\[
A_u^{\mathrm{ref}} \nRightarrow I_{t+k} = I_t \quad \text{for any } k > 0.
\]

Define expressive operators \(E \subset A\) with bounded codomain, corresponding to prefabricated representations such as avatars and stickers. These operators satisfy
\[
| \mathrm{Im}(E) | \ll | \mathrm{Im}(\text{natural language}) |.
\]

Let following be constrained by a finite capacity \(N_f\), while unsolicited content injection is unbounded:
\[
|F_t \setminus C_{\mathrm{followed}}| \not\leq M \quad \text{for any fixed } M.
\]

Define attention as a finite resource \(T_u\) allocated over \(F_t\). The ranking functional \(R\) is optimized to maximize interaction frequency
\[
\max_R \sum_t \mathbb{E}[\text{engagement}(F_t)]
\]
rather than coherence, duration, or convergence.

Under these assumptions, the system admits no equilibrium in which silence, low expressivity, or long-horizon curiosity are stable states.

\section{Constraint Specification}

Let time be indexed by \(t \in \mathbb{N}\).

Let \(F_t\) denote the feed presented to a user at time \(t\).

\[
F_t = C_t^{\mathrm{sel}} \cup C_t^{\mathrm{inj}}
\]

where \(C_t^{\mathrm{sel}}\) is content selected via explicit user actions and
\(C_t^{\mathrm{inj}}\) is content injected by the system.

\[
C_t^{\mathrm{inj}} \neq \varnothing \quad \forall t
\]

Define a dismissal operator \(d : F_t \to \{0,1\}\).

Assume
\[
d(c)=0 \nRightarrow c \notin F_{t+k} \quad \text{for any } k>0.
\]

Let expressive actions \(E\) be finite-valued:
\[
|E| < \infty
\]

Let non-expression be denoted by \(\varnothing_E\), and assume
\[
\varnothing_E \notin \arg\min R
\]
where \(R\) is the ranking functional.

Let following be bounded:
\[
|\mathcal{F}_u| \le N_f
\]
and blocking be unbounded:
\[
|\mathcal{B}_u| \to \infty.
\]

Let attention \(A_t\) be finite and additive:
\[
\sum_t A_t \le A_{\max}.
\]

Assume optimization objective:
\[
R = \arg\max \sum_t \mathbb{E}[\mathrm{interaction}(F_t)].
\]

Then no fixed point exists such that
\[
F_{t+k} = F_t \quad \text{under } \varnothing_E.
\]

\section{Entropic Formulation}

Let the system evolve in discrete time \(t \in \mathbb{N}\).

Let \(X_t\) denote the informational state presented to a user at time \(t\), modeled as a finite measure over symbolic tokens.

Define entropy
\[
S_t = - \sum_{x \in X_t} p_t(x)\log p_t(x).
\]

Let user action induce a conditional update
\[
X_{t+1} = \Psi(X_t \mid a_t)
\]
where \(a_t\) ranges over admissible actions, including non-action \(a_t = \varnothing\).

Assume the existence of an injection operator \(J\) such that
\[
X_{t+1} = \Psi(X_t \mid a_t) \cup J_t
\quad\text{with}\quad
J_t \neq \varnothing \;\; \forall t.
\]

Define expressive operators \(E\) as low-cardinality projections:
\[
E : X_t \to Y, \quad |Y| \ll |X_t|.
\]

Let silence correspond to the null projection \(\varnothing_E\), and assume
\[
\frac{\partial}{\partial t}\mathbb{E}[S_t \mid \varnothing_E] > 0.
\]

Let attention be a bounded resource \(A\) such that
\[
\sum_t \Delta A_t \le A_{\max}.
\]

Assume the system objective maximizes interaction rate rather than entropy minimization:
\[
\max \sum_t \mathbb{E}[\Delta I_t]
\quad\text{subject to}\quad
\frac{\partial S_t}{\partial t} \ge 0.
\]

Then the dynamics admit no steady state \(X^\ast\) satisfying
\[
X_{t+k} = X_t \quad \text{for } k>0
\]
under null action.

Instead, silence corresponds to an entropically unstable trajectory, and convergence requires sustained projection through finite expressive channels.


\end{document}
